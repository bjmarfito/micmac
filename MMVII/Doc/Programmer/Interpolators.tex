

%   ------------------------------------------------------------------
%   ------------------------------------------------------------------
%                 Chapter Interpolators
%   ------------------------------------------------------------------
%   ------------------------------------------------------------------

\chapter{Interpolators in \PPP}

\label{ChapInterpolators}

    % = = = = = = = = = = = = = = = = = = = = = = =
    % = = = = = = = = = = = = = = = = = = = = = = =
    % = = = = = = = = = = = = = = = = = = = = = = =

\section{Introduction}
This chapter groups the functionnality added in \PPP for image interpolation. In this
first draft different view-points are merged in the same chapter : theory, users, programmers,
and finally mainteners of the core-\PPP.  This organization may evolve with dispatching of different section
of this chapter in different part.

The following header  files , in {\tt include/} are involved in interpolators declaration :

\begin{itemize}
    \item {\tt MMVII\_Interpolators.h } : main file, contains the declaration of interpolators class;
    \item {\tt MMVII\_Images.h} ,{\tt MMVII\_Image2D.h} declaration of methods involving interpolators
          in images;

    \item {\tt MMVII\_TplSymbImage.h}  declaration of method using interpolators in image differentiation
          (as alternative to bilinear mode).
\end{itemize}

The following source  files , in {\tt src/} are involved in interpolators definition :

\begin{itemize}
   \item {\tt UtiMaths/Interpolators.cpp} definition of interpolator classes;
   \item {\tt ImagesBase/cIm2d\_Interpolators.cpp} definition of methods for interpolating $2-d$ images with
         \PPP's interpolators;
   \item {\tt Bench/BenchInterpolators.cpp}  definition of method for checking correctness of implementation
         (unitary test);

   \item {\tt Bench/BenchTutoImageDef.cpp} contains the test for image differenciation, initially made for bilinear mode,
         has been extended to take into account  interpolators.
\end{itemize}


    % = = = = = = = = = = = = = = = = = = = = = = =
    % = = = = = = = = = = = = = = = = = = = = = = =
    % = = = = = = = = = = = = = = = = = = = = = = =

\section{Theoreticall background}

\subsection{introducion}

Theory of image/signal interpolation is an essential  part of image processing and there is a huge 
documentation on the subject. The very brief theoreticall elements given here are more targeted to fix notation
than to be a complete summary of the theory.

In the more general case, in interpolation we have a function $F$  of $\RR^n$ sampled on a finite  set $p_k,v_k$
\footnote{or countable and isolated} set of point and we want to extend the value to whole
$\RR^n$  using some  property on $F$ :

\begin{itemize}
   \item $F$ is regular enough, whatever it means;
   \item $F(p_k) = v_k$ if we require a perfect match, or $F(p_k) \approx v_k$  if require only a close
         match,  whatever it means;
\end{itemize}



In the case of image/signal processing, we suppose that we know the value on regular grid $\{p_k\}=\ZZ^n$
and want to extand $F$ from $\ZZ^n$ to whole $\RR^n$ . 
Considering for know, a $1d$ interpolation, we have $p_k=k$, we note $\Interpol :  {v_k}  \rightarrow F$ 
the interpolation process.  

    % = = = = = = = = = = = = = = = = = = = = = = =

\subsection{$1d$ interpolation kernel}

We generally assume the following properties ,linearity as in~\ref{IntLin} :

\begin{equation}
    \Interpol_{\{a*v_k+b*w_k\}} = a*\Interpol_{\{v_k}\}+ b*\Interpol_{\{w_k\}} \label{IntLin}
\end{equation}

Invariance by translation, which mean that is we translate globally the $v_k$ of an integer $n$
we must translate the function of $n$ :

\begin{equation}
    \Interpol_{\{v_{k+n}\}}(x) =  \Interpol_{\{v_{k}\}}(x+n)  \label{IntTrans}
\end{equation}

The property of equation \ref{IntLin} and \ref{IntTrans}, have for consequence that interpolation
can be entirely caracterized by "kernel" function $\KernI : \RR \rightarrow \RR $ such that :

\begin{equation}
    F(x) =  \Interpol_{\{v_{k}\}}(x) = \sum_{k}  v_k  \KernI(x-k)   \label{KernIntDef}
\end{equation}

To assure the  integrity of interpolation process we generally assume properties on $\KernI$
such as $\KernI$ is continous (or differrentiable) and  $\int_{\RR} \KernI ^2 $ is defined. We dont
discuss in detail these "natural" hypothesis.
We generally assume a symtetry hypethosis on interpolation that can be traduced in symetry on
$\KernI$ :

\begin{equation}
    \KernI(-x) =   \KernI(x)   \label{KernIntSym}
\end{equation}


A "natural" property of interpolation is that if the value $v_k$ are constant $v_k=c$, the interpolated function 
is constant $F(x)=c$ . From the kernel point of view this property is traduced as the \emph{aka} partition of unity :

\begin{equation}
    \forall x \in \RR : \sum_k \KernI(x+k) =   1 \label{PartUnit}
\end{equation}


The constraint that the intepolation coincide with values on sample point ($F(k)=v_k$) if it
applies  is traduced by equation (where $\delta_x$ is kronecker's symbol) meaning that $ \KernI$ must
be null for any non null integer value, and must value $1$ for $k=0$ :

\begin{equation}
    \forall k \in \ZZ :  \KernI(k) =   \delta_0 \label{KernIntDelta0}
\end{equation}

For practicall reason, we cannot compute infinite sum like in equation~\ref{KernIntDef}, and we 
generally make the hypothesis that the support of $\KernI$ is bounded by a certain value $Sz_K$ :


\begin{equation}
    \forall x \in \RR   :  |x| > Sz_K  \Rightarrow \KernI(x) = 0 \label{KernIntBounded}
\end{equation}

    % = = = = = = = = = = = = = = = = = = = = = = =

\subsection{N-dimentional interpolation}

    %    -  -  -  -  -  -  -  -  -  -  -  -  -  -  -  -  -  -  -  -  -  -  -  -  -  -  -  -  -  -  -  -  -  -  -  -
\subsubsection{Separable kernels}

To keep things simple, we describe the interpolation on a $2$ dimensionnal grid $\ZZ^2$ wich is
usefull for image processing, but generalization to  higher dimension is obvious. For $N-d$
interpolation, we the same hypothesis than for $1d$ on linearity and invariance by translation we
can deduce the existance of a kernel function $\KernI(x,y)$ such that :

\begin{equation}
    F(x,y) =  \Interpol_{\{v_{i,j}\}}(x,y) = \sum_{i,k}  v_{i,j}  \KernI(x-i,y-j)   \label{KernInt2DDef}
\end{equation}

We generally make the separability hypothesis, this hypothesis says in general that interpolation
in $x,y$ can be done from the interpolation in $x$ of the sample interpolated in $y$. For a interpolation 
kernel , this is traduced as :

\begin{equation}
     \KernI(a,b)  = \KernI_x(a) \KernI_y(b)  \label{KernIntSep} 
\end{equation}

Where $\KernI_x$ and $\KernI_y$ are the interpolation kernels for $x$ and $y$. Also in general $x$ and $y$ play
the same role, which the case in image processing and we have :

\begin{equation}
      \KernI_x(a) =  \KernI_y(a)  \label{KernIntSymSep} 
\end{equation}

So finally we have the equation :

\begin{equation}
    F(x,y) =   \sum_{i,j}  v_{i,j}  \KernI(x-i) \KernI(y-j)   \label{KernInt2DDef}
\end{equation}

    %    -  -  -  -  -  -  -  -  -  -  -  -  -  -  -  -  -  -  -  -  -  -  -  -  -  -  -  -  -  -  -  -  -  -  -  -
\subsubsection{Synthesis}

We can make a brief synthesis of interpolation model we use in \PPP,  the interplation of an image $v_{i,j}$ can
be computed by following formula :

\begin{equation}
    F(x,y) =  \sum_{i,j}  v_{i,j}  \KernI(x-i) \KernI(y-j)   \label{IntIm2D}
\end{equation}


Where the kernel  $\KernI$ is a  $\RR \rightarrow \RR$ function with the following properties :

\begin{itemize}
    \item  $\KernI$ is regular (continous at least) and $\int_{\RR} \KernI^2$ is finite;
    \item  $\KernI$ is symetric (see \ref{KernIntSym});
    \item  $\KernI$ complies with partition of unity property (see \ref{PartUnit});
    \item  $\KernI$ is \emph{generally} such that $F(x,y)$ coincide with $v_{x,y}$ for integer
           values of $x,y$, which lead to  equation \ref{KernIntDelta0} ;
    \item  $\KernI$ is \emph{practically} bounded to an interval $Sz_K$ as in equation~\ref{KernIntBounded}.
\end{itemize}

    %    -  -  -  -  -  -  -  -  -  -  -  -  -  -  -  -  -  -  -  -  -  -  -  -  -  -  -  -  -  -  -  -  -  -  -  -
\subsubsection{Remark on implemantaion efficiency}

\label{InterpValueEff}

Obviously, in equation~\ref{IntIm2D} we can  restrict the sum to $i \in [x-Sz_K,x+Sz_k]$, and
idem for $j$. This said, a "naive" implementation of~\ref{IntIm2D}, can lead
to $2*(1+2*Sz_K)^2$ evaluation of $\KernI$. 

A basic optimization consist to use the separability and write :

\begin{equation}
    F(x,y) =  \sum_{j}  \KernI(y-j)   \sum_{i}  v_{i,j}  \KernI(x-i)  \label{IntIm2DOpt}
\end{equation}

Now if, before entering the main loop, we pre-compute in a table the value of $\KernI(x-i)$,  
we will just need to  compute once the $\KernI(y-j)$ and $\KernI(x-i)$

    % = = = = = = = = = = = = = = = = = = = = = = =

\subsection{Derivate the interpolation}

\label{InterpDeriv}

When the image is considered as function of $x,y \in \RR^2$, interpolation by kernel
offer not only the possibity to get the value for real coordinates, but also a \emph{natural}
and relatively efficient, way to compute the exact derivatives, provide that $\KernI$ is
a differentiable funcion. Noting :

\begin{equation}
       \KernI' =  \frac{\partial \KernI(x)}{\partial x}  \label{DefDerKern}
\end{equation}

Going back to equation \label{KernIntDef}, we have :


\begin{equation}
    F'(x)  = \sum_{k}  v_k  \KernI'(x-k)   \label{KernIntDeriv1D}
\end{equation}

Equation show that we can compute the derivative of interpolated function
with a formula similar to  interpolation where $\KernI'$ replace $\KernI$.
Let study then briefly the properties of $\KernI'$ when 
 $\KernI$ is a differentiable interpolation kernel :

\begin{itemize}
   \item  $\KernI'$ is bounded on the same support than $\KernI$;
   \item  $\KernI'$ is an odd function because $\KernI$ is an even function;
   \item  if $\KernI$ is a partition of unity (equation~\ref{PartUnit}) then $\KernI'$ complies with equation  ~\ref{PartZero};
\end{itemize}

\begin{equation}
    \forall x \in \RR : \sum_k \KernI'(x+k) =   0 \label{PartZero}
\end{equation}

For $2d$ interpolation, we have similarly :

\begin{equation}
    \frac{\partial F(x,y)}{\partial x} =  \sum_{i,j}  v_{i,j}  \KernI'(x-i) \KernI(y-j)   \label{Int2DerX}
\end{equation}

And  :
\begin{equation}
    \frac{\partial F(x,y)}{\partial y} =  \sum_{i,j}  v_{i,j}  \KernI(x-i) \KernI'(y-j)   \label{Int2DerY}
\end{equation}

Note also, that very often when need derivates, we need simultaneaously the $3$ values $F(x,y), \frac{\partial F(x,y)}{\partial x}$
and  $\frac{\partial F(x,y)}{\partial y}$, it is advantageaous then to compute the $3$ value in the same function, because
a shown by equations~\ref{InterpSimultComp} several computation can be shared :

\begin{itemize} 
    \item obviously share the loops on $i,j$;
    \item share the tabulation of $\KernI(y-j)$;
    \item share the computation of $\sum_{i}  v_{i,j}  \KernI(x-i)$.
\end{itemize} 

\begin{equation}
\left\{ \begin{array}{rcl}
     F(x,y)                                    &   \mbox{=} &  \sum_{j} \KernI(y-j)  \sum_{i}  v_{i,j}  \KernI(x-i)   \\
      \frac{\partial F(x,y)}{\partial x}       &   \mbox{=} &  \sum_{j} \KernI(y-j)  \sum_{i}  v_{i,j}  \KernI'(x-i)  \\
      \frac{\partial F(x,y)}{\partial y}       &   \mbox{=} &  \sum_{j} \KernI'(y-j) \sum_{i}  v_{i,j}  \KernI(x-i)  
\end{array}\right.
\label{InterpSimultComp}
\end{equation}

    % = = = = = = = = = = = = = = = = = = = = = = =
\subsection{Standard interpolation kernel used in \PPP}

All the standard intepolator complies with equation~\ref{KernIntDelta0}.

   %   -  -  -  -  -  -  -  -  -  -  -  -  -  -  -  -  -  -  -  -  -  -  -  -  -  -  -  -  -  -  -  -  -  -  -  -
\subsubsection{Nearest neighboor interpolation kernel}

\emph{Not implemented for now} because it has very poor property (not even continous), but may be added 
(for test/comparison ?).

The kernel has support $S=]-\frac{1}{2},\frac{1}{2}]$ and we have $\KernI(x)=1$/

   %   -  -  -  -  -  -  -  -  -  -  -  -  -  -  -  -  -  -  -  -  -  -  -  -  -  -  -  -  -  -  -  -  -  -  -  -
\subsubsection{Linear interpolation kernel}
\label{LinearInterp}

The linear interpolator has support $[-1,1]$ and is defined by the following kernel :

\begin{equation}
  \KernI(x)= |1-x|
\end{equation}

It is continous but not derivable.


   %   -  -  -  -  -  -  -  -  -  -  -  -  -  -  -  -  -  -  -  -  -  -  -  -  -  -  -  -  -  -  -  -  -  -  -  -
\subsubsection{Sinus Cardinal interpolation kernel}
\label{SinCInterp}


In it's  "pure" definition it's an unbounded interpolator defined by :

\begin{equation}
  \KernI(x)= \sinc(\pi x) = \frac{\sin(\pi x)}{\pi x}
\end{equation}

It's $C_{\infty}$ and is interesting from the theoreticall point of view as it allows
to recover exactly the  function from its sampling if the function complies with \emph{Nyquist-Shannon}
hypothesis. Practically no realistic function complies with \emph{Nyquist-Shannon},   but
it still remain interesting as a model when we require the "best ressampling possible".

The function $\sinc$  veririfies equation  \ref{PartUnit} (reference for this non obvious property ??).


   %   -  -  -  -  -  -  -  -  -  -  -  -  -  -  -  -  -  -  -  -  -  -  -  -  -  -  -  -  -  -  -  -  -  -  -  -
\subsubsection{Apodized Sinus Cardinal interpolation kernel}
\label{SinCApodInterp}


A practicle limitation of $\sinc$-kernel is its infinite support that makes it impossible to use as is.  
By the way, it remains an interesting option for these reconstruction aspect.  Also, "cuting" brutally
the $\sinc$ to have a bounded kernel is not a good idea as it would creat high frequency, so generally
we use an "apodization" function $A$ that generate a smooth transition, typically we will have :

\begin{itemize}
    \item $A(x)=1$  for $x \in [0,a] $
    \item $A(x)$  decrease continuoussly from $1$ to $0$ for  $x \in [a,a+b] $, a basic example is
          $A(x) = L_{a,b}(x) = \frac{b-x}{b-a}$, but can take also a smoother version as $Cub_0 \circ L_{a,b}$
          where $Cub_0$ is the cubic function of parameter $0$ (see \ref{CubicInterpol});
\end{itemize}

The  $\sinc$ apodized is then defined by :

\begin{equation}
  \KernI(x)  = A(x) \sinc(\pi x)  \label{SinCApod}
\end{equation}

The formula cannot be used as is, because if $\sinc$ complies with~\ref{PartUnit}, it's no
more the case of $\sinc * A$, an eathy way to overcome this issue (and work for any kernel) is
to replace by the normalized kernal $\KernI^n$:

\begin{equation}
  \KernI^n(x)  =  \frac{\KernI(x)}{\sum_{k \in \ZZ} \KernI(x+k)} \label{InterNormKern}
\end{equation}

However we have then $2$ other issue :

\begin{itemize}
   \item  computation of formula~\ref{InterNormKern} can be a time-consuming ;
   \item  derivation of formula~\ref{InterNormKern}, as required in~\ref{InterpDeriv} 
          for dervivation of intepolated function, can become quite complex.
\end{itemize}

Concretely, these issues can be addressed efficiently  using a \emph{tabulated} implementation
of the apodized $\sinc$ as described in~\ref{InterpolTabul}.


   %   -  -  -  -  -  -  -  -  -  -  -  -  -  -  -  -  -  -  -  -  -  -  -  -  -  -  -  -  -  -  -  -  -  -  -  -

\subsubsection{Cubic interpolation kernel}

\label{CubicInterpol}



Cubic intepolation is a current choice as a compromise "quality/efficiency" : it is derivable and has a kernel of $2$ .
Each cubic interpolator if define by a parameter $p$ which is the value of derivate in $1$, we have :

\begin{equation}
\KernI(x) = Cub_p(x)
\left\{ \begin{array}{rcl}
(p+2 ) x^3 -(p+3)x^2+1        &   \mbox{for} &  0 \leq x \leq 1 \\ 
p*(x^3 - 5 * x^2 + 8* x -4)   &   \mbox{for} &  1 \leq x \leq 2 \\
0                             &   \mbox{for} &  2 \leq x \\
\KernI(-x)                    &   \mbox{for} &   x \leq 0 \\
\end{array}\right.
\end{equation}


The cubic interpolator has following properties :

\begin{itemize}
    \item  it is derivable, it suffice to check that $\frac{\partial Cub_p(x^-)}{\partial x} = \frac{\partial Cub_p(x ^+)}{\partial x} $  
           for $x \in \{0,1,2\}$;
    \item  it complies with proterties \ref{KernIntBounded},   \ref{PartUnit} and \ref{KernIntDelta0}  
\end{itemize}


For theoreticall reason (which ??) , the parameter $p$ should be in $[-3,0]$ . Some remarq on parameter $p$ :

\begin{itemize}
    \item  when used in image ressampling, the higher value of  $p$ enhance the high frequency of images, which can be used
           for zooming (upscaling) images ;

    \item  for  $p=0$  the support is $[0,1]$ and $\KernI(x) \geq 0$ , it can be used for image downscaling,
           it can be seen as an approximation of gaussian with small support;

    \item  for  $p= -0.5$  the interpolator  is such that interpolation of purely linear images (as $I(x,y)=a+bx+cy$)
           will  be linear; in some way, it will be the default choice when 
\end{itemize}

    % = = = = = = = = = = = = = = = = = = = = = = =
\subsection{Non Standard interpolation kernel used in \PPP}

\label{MMVIIInterpol}

The origin of these interpolator go back to the analysis of existing bias when using interpolation
in image correlation : in a context where the phase should theoretically be uniformly 
distributed in $[0,0.5]$ we  observe that some phase are "privilegied"  (?? which one, experience to be done ??).

The bias is particularly important with bilinear interpolator, a possible
analysis of these fact is the following :

\begin{itemize}
    \item  the "bluring" effect of the interpolator varies with the phase;
    \item  if the phase is $0$, the pixel will be a single pixel value 
    \item  if the phase is $\frac{1}{2}$, the pixel will be the average of two
           pixel, having a slight blurring effect.
\end{itemize}

To try to overcome this problem, we use the following reasonning :

\begin{itemize}
    \item  for phase  $\frac{1}{2}$, there is no much more to do than to have
           a weigting $0.5,0.5$ for the $2$ nearest neighboor if we want to limit the number of neigboors
           involved;
    \item  for phase  in $x \in [0,0.5]$ , we want to limit the size of kernel,
           but need to add at least a third pixel if we want to have weightin
           that is centered on $x$ and has the same bluring effect than for phase $\frac{1}{2}$;
    \item  regarding the blurring effect, it is caracterized as the average of the power $\alpha$ of 
           to the center distance , where $\alpha$ is parameter of the kernel (for $\alpha=2$ it's the
           variance);
     \item let's name $a,b,c$ the weighting of pixels $-1$, $0$ and $1$ for phase $x \in [0,0.5]$, the value
           of $a,b$ and $c$ can be evaluated by equations~\ref{KernInMMVIIEq}
\end{itemize}

\begin{equation}
\left\{ \begin{array}{rc|l}
  a+b+c        &    = 1  &  \textrm{it's a weigthing}  \\ 
  -a+c      &       = x &   \textrm{center is on } x \\
  a(1+x)^\alpha        + b x^\alpha +c(1-x)^\alpha & = \frac{1}{2} ^{\alpha} &  \textrm{maintain } \alpha  \textrm{-average of distance}
\end{array}\right.
\label{KernInMMVIIEq}
\end{equation}

For $\alpha=2$, the equations~\ref{KernInMMVIIEq} lead to a simple analyticall formula for the kernel :

    % if (anX<=0.5)  return 0.5 *(1.5-2*Square(anX));
    % if (anX<=1.5)  return 0.5 * Square(anX-1.5) ;

\begin{equation}
\KernI(x) = M_2(x)
\left\{ \begin{array}{rcl}
\frac{1.5-2x^2}{2}       &   \mbox{for} &  0 \leq |x| \leq \frac{1}{2} \\ 
\frac{(x-1.5)^2}{2}       &   \mbox{for} &  \leq \frac{1}{2} \leq |x| \leq \frac{3}{2} \\ 
0                             &   \mbox{for} &  |x| \geq \frac{3}{2}
\end{array}\right.
\label{KernInMMVII2Eq}
\end{equation}

For other value of $\alpha$ , the expression is not so simple, by the way if we need to use/test 
different value, we proceed this way :

\begin{itemize}
     \item we have a numerical version of the parametrized that for each phase solve the system 
           equations given in~\ref{KernInMMVIIEq};
     \item this numerical version  is relatively slow, but it's not a problem if we use the
           tabulated version as described in~\ref{InterpolTabul}.
\end{itemize}


    % = = = = = = = = = = = = = = = = = = = = = = =
\subsection{Tabulated interpolators}
\label{InterpolTabul}


As interpolation  kernel are $1$ dimensionnal bounded function, their value can be easily tabulated, this
means :

\begin{itemize}
    \item  fix a number $Nb$  of value per unity ;
    \item  create a tab $T$ of size $Nb*Sz_K$ ;
    \item for $K \in [0,Nb*Sz_K]$  set  $ T[K] = \KernI(\frac{K}{Nb})$
    \item when required to compute $\KernI(x)$  we simply extract the value of $T[x*Nb]$,
          preferably using bilinear interpolation.
\end{itemize}

Let note $T_\KernI$  the interpolator obtained by reading the values  $T[x*Nb]$.

First note that this technique can be accurate for relatively small cost in memory and accuracy:

\begin{itemize}
     \item typically consider the apodidzed sinus carninal with a kernel of $5$ and an apodiszation of $5$, tabulated with $Nb=1000$;
     \item the  cost in memory is $10000=(5+5)*1000$ elements, the cost of pre-computation is also $10000$  value;
     \item the  accuracy, i.e max difference between $\KernI$ and the tabulation value can be estimate to $\frac{1}{Nb^2} = 10^{-6}$
           (classical formula for accuracy of linear interpolation).
\end{itemize}

Having seen that the lost is low, see now what is the gain :

\begin{itemize}
    \item first obvious gain is time of computation of each new value ;

    \item second gain is that, once all the value has been stored in the tab, it's possible to do a post-processing
          using formula  similar to~\ref{InterNormKern} on the tabulated value; so finnaly $T_\KernI$ will 
          complies with unity partition formula (\ref{PartUnit});

    \item finnaly its possible to use $T$ for computing approximate values of derivates using formula
          like  $\frac{\partial T_\KernI}{\partial x} \approx \frac{T[k+1]-T[k-1]}{2*Nb}$, so we can
          make $T_\KernI$ differentiable, even if no analytical derivate is furnished for $\KernI$
          (but obvioulsy if $\KernI$ is intrinsically not differentiable, like linear interpolator, the
           results will be meanignless).
\end{itemize}

    % = = = = = = = = = = = = = = = = = = = = = = =
    % = = = = = = = = = = = = = = = = = = = = = = =
    % = = = = = = = = = = = = = = = = = = = = = = =

\section{Choice of an Interpolator}

    % = = = = = = = = = = = = = = = = = = = = = = =

\subsection{Interpolator for image up-izing}

Interpolator are mainly usefull for image resampling.  When the resampling is done
to a higher resolution, or a resolution equivalent the choice of an interpolator,
there is no so much to say, and it is mainly an affair of trade-off between accuracy and efficiency :

\begin{itemize}
   \item for fast computation and low accuracry, bi-linear can be a good choice;
         it's not recommanded when derivative are required;

   \item bi-cubic can be a good compromize quality/efficiency, the default recommander
         parameter being $P=-\frac{1}{2}$, but higher value can be used for image enhancing
         when used at high zooming;

   \item for "best" ressampling, with minimal aliazing, according to signal theory, the sinus cardinal
         can be used;

   \item finally, in image correlation the non standard interpolator \ref{MMVIIInterpol}, especially $M_2$,
         can be interesting compromize.

\end{itemize}

    % = = = = = = = = = = = = = = = = = = = = = = =

\subsection{Interpolator for image downsizing}

   %   -  -  -  -  -  -  -  -  -  -  -  -  -  -  -  -  -  -  -  -  -  -  -  -  -  -  -  -  -  -  -  -  -  -  -  -

\subsubsection{No aliasing property}

In image downsizing, we can still use interpolator and formula~\ref{KernIntDef},
however we need to take some precaution to  avoid aliazing. We make the analyis
in $1d$, the generalization to $Nd$ being direct.

When computing an image $I_S$ from an image $I$ with a  
down sizing of size $S$, with a kernem $\KernI_S$, we write formula~\ref{KernIntDef}  this way :

\begin{equation}
     I_S(k') =   \sum_{k}  I(k)  \KernI_S(S k'-k)   
\end{equation}

To avoid aliasing, we must take care that each $v_k$ contributes equally to the final result :

\begin{equation}
    \sum_{k'}  \KernI_S(S k'-k)  = 1 \label{Interp:DS1}
\end{equation}

If we consider an initial kernel $\KernI$ and define :

\begin{equation}
    \KernI_S(x) = \frac{\KernI(\frac{x}{S})}{S}
\end{equation}


Then property~\ref{Interp:DS1} is  satistied if initial kernel $\KernI$ is
a partition of unity :

\begin{equation}
    \sum_{k'}  \KernI_S(S k'-k)  =  \sum_{k'}  \KernI( k'-\frac{k}{S})  = 1
\end{equation}

All interpolator used in \PPP are partition of unity (at least once tabulated).
For downsizing is seems  \emph{"Natural"} to select an interpolator with only positive
coefficient, it we want to take as model the physcical sensor.  An also a smooth
function seems preferable :

\begin{itemize}
    \item $SinC$ and cubic for  $p \neq 0$ are non postive interpolator ;
    \item  nearest neighboor is certainly not smooth;
    \item  linear interpolator, cubic interpolator for $p=0$ are continous interporlator;
    \item  $M_2$ interpolator can also be used, it's draw back is to be non standard,
           and maybe to have a slight blurring effect
\end{itemize}


   %   -  -  -  -  -  -  -  -  -  -  -  -  -  -  -  -  -  -  -  -  -  -  -  -  -  -  -  -  -  -  -  -  -  -  -  -

\subsubsection{"No spatial bias" property}

Someway,  cubic interpolator  with $p=0$, may seems the best choice : smooth (differentiable), positive and support
reduced to $[-1,1]$.  But there  is another criteria that we need to take care.   We want that the
downsampling does not create any spatial bias. By that we define the centroid $C(I)$ of $I$
as  :

\begin{equation}
    C(I)  =  \frac{\sum_k k I(k)} {\sum_k I(k)}
\end{equation}

Considering that $I$ is already normalised such $\sum_k I(k)=1$, we have
$C(I) = \sum_k k I(k)$. By no spatial bias, we mean formally :


\begin{equation}
    C(I_S)  =  \frac{C(I)}{S} \label{Interp:Eq:Centroid}
\end{equation}

Due to linearity, it is sufficient that equation \ref{Interp:Eq:Centroid} is satisfied
for functions kroneckers's function $\delta^x$. We obviously have $C(\delta^x) = x$.
We have :

\begin{equation}
    \delta^x_S(k') = \frac{1}{S} *  \sum_{k}  \delta_x(k)  \KernI_S(S k'-k ) =  \frac{1}{S} \KernI_S(S k'-x)
\end{equation}

So we have :

\begin{equation}
    C(\delta^x_S) =  \sum_{k'}  \KernI(k'-\frac{x}{S}) k'  
\end{equation}


So we finnally have the equation for non spatially biased interpolator :

\begin{equation}
   \forall x     \sum_{k'}  \KernI(k'-\frac{x}{S}) k'  = \frac{x}{S} \label{Interp:NoSpatialBias}
\end{equation}


The function {\tt TestBiasInterpolator} of \PPP make an experimental test on formula
\ref{Interp:NoSpatialBias}. The results of this test is the following :

\begin{itemize}
    \item  linear, $SinC$ and   $M_2$  are not spatially biased;

    \item  cubic interpolator for $p=-0.5$ is not spatially biased (this is the interpolator that
           transformat lines in lines);

    \item  other cubic interpolator are spatially biased .
\end{itemize}

So finally, the cubic interpolator is not such a goof choice for image down sampling, 
because the only  non negative  is biased.  This used to be the default choice for
image down sampling, but it's no longer the case. For now the default is the linear
interpolator, maybe coud be replaced by $M_2$ ?


    % = = = = = = = = = = = = = = = = = = = = = = =
    % = = = = = = = = = = = = = = = = = = = = = = =
    % = = = = = = = = = = = = = = = = = = = = = = =

\section{Users's view}

\label{InterpUserView}

Once the theoreticall presentation of interpolator is made, from user's view the point is
how to specify an interpolator for the commands that takes it as a parameter.
The user will specify an interpolator as a  vector of string, this vector can be :

\begin{itemize}
    \item {\tt [Linear]}  for creating a linear interpolator (see \ref{LinearInterp})
    \item {\tt [MMVII]}  for creating the non standard interpolator of parameter $2$, which more of less the default value 
                         (see~\ref{MMVIIInterpol});
    \item {\tt [Cubic,Param]} for creating a cubic interpolator of parameter {\tt Param}, where {\tt Param} is the value
          of derivate in $1$ (see  \ref{CubicInterpol});
    \item {\tt [MMVIIK,Param]} for creating a non standard intepolator with exponent {\tt Param} 
                         (see~\ref{MMVIIInterpol});
    \item {\tt [SinCApod,Param1,aParam2]}  for creating an apodized sinus cardinal interpolator, which value exactly
         $\sinc$ for    $ |x| \leq P_1$  , and is apodized for   $ P_1\leq |x| \leq P_1+P_2$
                         (see~\ref{SinCApodInterp});
    \item {\tt [Tabul,Nb, \dots ]} for creating a tabulated interpolator , wher  $Nb$ is the number of value per unit  
          and {\tt \dots} must describe the interpolator to be tabulated
\end{itemize}

Here are some example of valid vector of strings :

\begin{itemize}
    \item   {\tt [Cubic,-0.5]} ;
    \item   {\tt [Tabul,1000,MMVIIK,1.0]};
    \item   {\tt [Tabul,1000,SinCApod,5.0,5.0]} ;
\end{itemize}

Note that it is not recommanded (when not forbiden) to create non tabulated version of {\tt MMVIIK} and 
{\tt SinCApod}.

    % = = = = = = = = = = = = = = = = = = = = = = =
    % = = = = = = = = = = = = = = = = = = = = = = =
    % = = = = = = = = = = = = = = = = = = = = = = =

\section{Progammer's view}

This section contains the documentation for the programmer that want to use, \emph{as is}, the
interpolators library of {\tt MMVII} but does not need to make it evolve (i.e correct bug, add new interpolators ,
accelerate \dots).


    % = = = = = = = = = = = = = = = = = = = = = = =

\subsection{Interpolator classes}

The declaration of interpolators classes can be found in {\tt include/MMVII\_Interpolators.h } .
The interpolator classes will derive of {\tt cInterpolator1D}  or, if it is differentiable,
{\tt cDiffInterpolator1D}.   An interpolator essentialy describe its interpolation kernels.

Also interpolators will be probably mostly used at {\tt macro}
level with  methods described in~\ref{InterpolProgImAccess}, elementary methods
of {\tt cInterpolator1D}  and {\tt cDiffInterpolator1D}
can be used for a \emph{fine} specific usage . For {\tt cInterpolator1D} :

\begin{itemize}
   \item  {\tt tREAL8 SzKernel() const;}  access to the size of the kernel;

   \item  {\tt virtual tREAL8  Weight(tREAL8  anX) const = 0;} "fundamental" method, return
          for each phase the value of the kernel $\KernI$  function, it's a pure virtual method
          which will be overided in concrete derivate class;

    \item {\tt const std::vector<std::string> \& VNames() const ;} return a vector of string
          describing the intepolator, not sure very usefull , maybe for generating message of error ?
\end{itemize}

For {\tt cDiffInterpolator1D} :

\begin{itemize}
    \item   {\tt virtual tREAL8  DiffWeight(tREAL8  anX) const =0;}  return the value of $\KernI'(x)$;

    \item {\tt virtual std::pair<tREAL8,tREAL8>  WAndDiff(tREAL8  anX) const ;}  return a pair containing
          $\{\KernI(x),\KernI'(x)\}$, it these $2$ value are required it may be more efficient than calling
          successively {\tt Weight} and {\tt DiffWeight}.
\end{itemize}

An interpolator can be created directly using one the derivate class implementing "analyticall"
interpolators :  {\tt cLinearInterpolator}, {\tt cCubicInterpolator}, {\tt cSinCApodInterpolator},
{\tt cMMVII2Inperpol}, {\tt cMMVIIKInterpol}.


A tabulated interpolator can be created using class {\tt cTabulatedDiffInterpolator},
the constructor take as parameter an interpolator and the number of tabulation.
The class {\tt cTabulatedInterpolator} is non differentiable, it's rather an implementation
class and is not described here (see~\ref{InterpolMaintener}).


To create an interpolator using the vector of string as described in~\ref{InterpUserView},
the {\tt static}  method {\tt AllocFromNames} of class {\tt cDiffInterpolator1D} can be used.
The object returned is of type {\tt cDiffInterpolator1D *} and must be deleted to avoid
a memory check error.

    % = = = = = = = = = = = = = = = = = = = = = = =

\subsection{Images intepolation}
\label{InterpolProgImAccess}

The method for doing interpolation of images are accessible as method of  the class {\tt cDataIm2D}.


   %   -  -  -  -  -  -  -  -  -  -  -  -  -  -  -  -  -  -  -  -  -  -  -  -  -  -  -  -  -  -  -  -  -  -  -  -

\subsubsection{{\tt InsideInterpolator}}

{\tt bool InsideInterpolator(const cInterpolator1D \& anInt,const cPtxd<double,Dim> \& aP,tREAL8 aMargin=0.0) const; }

The method is defined in class {\tt cPixBox}, and is accessible from class  {\tt cDataIm2D<Type>}
as it inherits from {\tt cPixBox<2>}.

This method indicate if  interpolator {\tt anInt} can be used with pixel {\tt P} : i.e. if {\tt P}
is \emph{"sufficiently"} inside the image taking account the size of the kernel (and an optionnal
margin if we want it even more inside).

   %   -  -  -  -  -  -  -  -  -  -  -  -  -  -  -  -  -  -  -  -  -  -  -  -  -  -  -  -  -  -  -  -  -  -  -  -

\subsubsection{{\tt GetValueInterpol}}

{tREAL8 GetValueInterpol(const cPt2dr \& aP,const cInterpolator1D \& anInt) const;}

The method return the value of image at point {\tt aP} using interpolator {\tt anInt}. 
The computation is done using optimisation described in~\ref{InterpValueEff}.
In debug mode, an error will occurs if the point is not \emph{"sufficiently"} inside 
the image. So if there is any doubt, a test of {\tt InsideInterpolator} will be required before calling it.

   %   -  -  -  -  -  -  -  -  -  -  -  -  -  -  -  -  -  -  -  -  -  -  -  -  -  -  -  -  -  -  -  -  -  -  -  -

\subsubsection{{\tt GetValueAndGradInterpol}}

\label{Method:GetValueAndGradInterpol}

{\tt std::pair<tREAL8,cPt2dr> GetValueAndGradInterpol(const cDiffInterpolator1D \&,const cPt2dr \& aP) const ;}

The method is similar to {\tt GetValueInterpol}, but instead of returning only the values, it 
return a pair containing the value (real) and the gradient (point).
The computation is done using optimisation describes in equation~\ref{InterpSimultComp}.


    % = = = = = = = = = = = = = = = = = = = = = = =

\subsection{Usage of interpolators}


We give a brieve description of possible usage of interpolators.

   %   -  -  -  -  -  -  -  -  -  -  -  -  -  -  -  -  -  -  -  -  -  -  -  -  -  -  -  -  -  -  -  -  -  -  -  -

\subsubsection{Image ressampling}

The classical usage of interpolators , nothing special to say, can be used for computing explicitely 
an image transformation or "on the fly" for example in image correlation.

   %   -  -  -  -  -  -  -  -  -  -  -  -  -  -  -  -  -  -  -  -  -  -  -  -  -  -  -  -  -  -  -  -  -  -  -  -

\subsubsection{Interpolator and gradient}

The method {\tt GetValueAndGradInterpol} can be used for computing the gradient in place
of other classical algorithm (sobel, deriche \dots), don't know the value of it, by the way
can give it a try.

Perhaps more interesting is when we want to compute simultaneously a geometric transformation $\phi$
and the gradient of the transformed image, if may be interesting for each pixel $p$ to :

\begin{itemize}
   \item computing simultaneously $I(\phi(p))$ and its gradient;
   \item use the jacobian on $\phi(p)$ for infering the gradient in transformed image.
\end{itemize}

   %   -  -  -  -  -  -  -  -  -  -  -  -  -  -  -  -  -  -  -  -  -  -  -  -  -  -  -  -  -  -  -  -  -  -  -  -

\subsubsection{Interpolator and images optimization}

When image are use in non linear optmisation (see~\ref{ImageOptDiff}), it is necessary to consider it like a
continuous function $\RR^2 \rightarrow \RR$. In this case there is two option in
\PPP (see \ref{SampleImageDiff}) :

\begin{itemize}
     \item  create an explicit interpolation function on formulas, see ~\ref{InterpBilCase};
     \item  make a local linear approximation (see~\ref{ImDifGradMode}), in this case the
            method {\tt GetValueAndGradInterpol} can be on option.
\end{itemize}

    % = = = = = = = = = = = = = = = = = = = = = = =
    % = = = = = = = = = = = = = = = = = = = = = = =
    % = = = = = = = = = = = = = = = = = = = = = = =

\section{Maintener's view}

By maintener's we mean personn that would : make bug correction or make evolve the library .

\label{InterpolMaintener}

    % = = = = = = = = = = = = = = = = = = = = = = =

\subsection{Evolution}

The following evolution can be envisaged :

\begin{itemize}
    \item \emph{add a new interpolator}, in this case it is recommanded to study the bench library describe
	    bellow;
    \item \emph{Efficiency} the main cost of computation are probably the internal loop  that compute
	    the  scalar produtc between a line of  image and the coeffx in x (search {\tt SCALARPRODUCT}
		in function {\tt GetValueInterpol, GetValueAndGradInterpol}

    \item \emph{Efficiency} a minor enhancement would be to add a method that compute not a single
	   weight, but, for a given phase $Ph$ compute the vector of weight for $Ph-k,Ph-k+1,\dots Ph+k$,
           for tabulated interpolator some computation would be shared;

   \item \emph{Partially inside} the fact that all the point must be inside the image with the sz of the kernel
	   may be too restrictive, especially with high kernel, maybe write a method for points that are not
		enough inside image.
\end{itemize}

    % = = = = = = = = = = = = = = = = = = = = = = =

\subsection{Bench for interpolator}

Several "bench" method have been written to check the correctness of interpolation. When making evolve the
libray, they should be at least run again to check that there is no regression, and eventually completed,
for example when adding new interpolator.

\begin{itemize}
     \item {\tt TplInterpol\_CmpLinearGetVBL}
     \item {\tt TplInterpol\_CmpCubicFoncLinear}
     \item {\tt TplInterpol\_FuncLowFreq}
     \item {\tt BenchIntrinsiqOneInterpol}
     \item {\tt BenchIntrinsiqMMVIIInterpol}
\end{itemize}


%% GetVBl +ValInterpol





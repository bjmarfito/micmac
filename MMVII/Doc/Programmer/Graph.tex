

%   ------------------------------------------------------------------
%   ------------------------------------------------------------------
%                 Chapter set editing
%   ------------------------------------------------------------------
%   ------------------------------------------------------------------

\chapter{Graph library in MMVII}


%   ------------------------------------------------------------------

\section{Introduction}

A  graph-libray (basic, for now) is available in \PPP.  The main objects that a user can
manipulate are, quite obviously: vertices, edges and graphs, the library
furnishes classes for those $3$ object, say $V,E,G$.
There also exist classes for algoritmic analyses (for example, the  shortest past).

To allow the user to associate any data they need with these object, the classes
are templatized by $3$ parameters defining the attributes of objects of the graph:

\begin{itemize}
      \item  class for attibute of vertices, denoted as $\AttrV$
      \item  class for attibute of oriented edges, denoted as $\AttrEOr$;
      \item  class for attibute of symetric (\emph{non} oriented) edges, denoted $\AttrESym$ (we will detail the difference
             between  $A_{\vec{e}}$ and  $A_{\overline{E}}$ in section~\ref{Gr:Attr:SymOr}).
\end{itemize}

When the user will create a graph $g$, they will specify the three types of attributes and create
an object of type $\ObjGr{G}$.  Then, when the user will create
vertices $v$ of $g$, these objects will be of type $\ObjGr{V}$;
similarly, the edges created and manipulated  will be of type $\ObjGr{E}$.

The code can be found in :

\begin{itemize}
    \item {\tt include/MMVII\_Tpl\_GraphStruct.h } for definition of structure for representing the graph;
    \item {\tt include/MMVII\_Tpl\_GraphAlgo\_SPCC.h } for basic graph algoritms : shortest path, minimum
          spaning tree and connected component.
\end{itemize}

Note that this is a "header only"  library, which is not the default choice in \PPP. However, providing a header-only library ensures its generality, thus it seemed an appropriate choice.

There also exists a file {\tt src/Graphs/BenchGraph.cpp} that contains (almost) extensive tests/benchmarks of the library.
It is densely commented, and it should serve as a detailed tutorial. Consequently, this chapter will be relatively
brief, and we invite users to deep-dive in the library by examining the file {\tt BenchGraph.cpp} .

%   ------------------------------------------------------------------
%   ------------------------------------------------------------------
%   ------------------------------------------------------------------

\section{The graph structures}

%   ------------------------------------------------------------------

\subsection{Creating objects}

Creating object is done in $3$ steps  :

\begin{itemize}
    \item  create a graph $g$ of type  $\ObjGr{G}$, this is done
           without any parameter;

    \item  create vertices  of $g$,  this is done by calling the method {\tt NewSom} of $g$,
          this method takes as parameter an attribute of type $\AttrV$;

    \item  create edges  of  $g$,  this is done by calling the method {\tt AddEdge} of $g$,
           if takes $5$  parameters :

    \begin{itemize}
          \item two vertices $v_1$ ans $v_2$  of type $\ObjGr{V}$;
          \item two oriented attributes of type $\AttrEOr$, one for edge $\GrEdgAB$
                and one for  $\GrEdgBA$
          \item one symetric attributes of type $\AttrESym$.
    \end{itemize}
\end{itemize}

%   ------------------------------------------------------------------

\subsection{Attributes and object copy, destruction}

The three kinds of objects $G,V,E$ cannot be copied. When the user gets access
to vertices and edges  using methods of the libray, they always manipulate pointers or references
on them. 

For the destruction of objects, it is necessary and sufficient to destroy $G$, 
which will automatically destroy all the $V,E$ created.  Obviously,  user must never
explicitely destroy $V$ or $E$.

After the creation of $V$ and $E$, the object will contain attributes  of type $\AttrV$, $\AttrEOr$ and $\AttrESym$;
all these attributes will be copies of the value given by the user.


%   ------------------------------------------------------------------
\subsection{Oriented and symetric attribute,}

\label{Gr:Attr:SymOr}

Algoritmic graph theory often distinguishes oriented from non-oriented graph.  In \PPP~ 
such distinction does not exist, and we do not use separate graphs types for it.  The same
graph can be considered as oriented or non-oriented. In practice, this distinction is made through 
the way the graph is used and through algorithms.

This is why we have two type of attribute for the edges : 

\begin{itemize}
   \item first, for each pair of connected vertices, there exist an edge  $\GrEdgAB$ and
         and an edge $\GrEdgBA$;

   \item the oriented attribute is \emph{physically} different for  $\GrEdgAB$
         and $\GrEdgBA$,  this is why we give $2$ values at creation;  if we
         modif the attribute of  $\GrEdgAB$, it will do nothing to  $\GrEdgBA$'s;

   \item conversely, the symetric attribute is   shared between the two edges, it exist
         \emph{physically} at a single location in memory,  and takes a single value at creation.
  
\end{itemize}

The distinction between oriented and non-oriented graphs will be done when runing algorithms and through their
parametrisation (see~\ref{Graph:AlgoBasic}). Typically, in the computation of the shortest path on a road network
with one way roads, the parametrisation will use $\AttrEOr$; while when computing the minimum spaning tree, 
the parametrization will use  $\AttrESym$.


%   ------------------------------------------------------------------
\subsection{Acces to object}

We don't give all the details, but the main accesses offered by the library are :

\begin{itemize}
    \item for a graph, acces to its vertices;

    \item for a vertex $v_1$, access to its attributes, and all its adjacent edges
          (all the edges of type  $\GrEdgAB$);

    \item for a pair of vertices $v_1,v_2$, knowing if the edge $\GrEdgAB$ exist,
          and then the acces to it;

    \item for an edge, acces to its oriented and non oriented attributes, access to the vertex it is pointing to (i.e. $v_2$).
\end{itemize}

%$A_V,\vec{A}_e,\overline{A}_E$,


%   ------------------------------------------------------------------
\subsection{Access to {\tt DirInit}}

Finally, to be complete, a tiny complex detail, when manipulating the symmetric
attribute of an edge $v_1v_2$, it may be interesting to know if the edge  was  created as $\GrEdgAB$
or $\GrEdgBA$. This is accessible via the method {\tt DirInit}.  See the (to
be written) section of graph on group, to see a concrete example.


%   ------------------------------------------------------------------
%   ------------------------------------------------------------------
%   ------------------------------------------------------------------

\section{The basic algorithms}
\label{Graph:AlgoBasic}

%   ------------------------------------------------------------------

\subsection{Parametrization of algorithms}

When graphs are used for representing data extracted from image analysis, for example a road
network in a satellite image, it often happens that we want to use the \emph{same} graph 
but in different variants, for example :

\begin{itemize}
    \item  consider only a subset of edges or vertices in connected components analysis;
    \item  consider different weighting for the shortest path computation.
\end{itemize}

To offer such flexibility, the algorithm of \PPP~ takes two kinds of parameters as object
deriving from base classes :

\begin{itemize}
    \item   {\tt cAlgo\_SubGr} parametrize a sub-graph, i.e. a subset of vertices and edges,
            this is done by the \emph{virtual} methods {\tt bool InsideVertex()} and {\tt bool InsideEdge()};
            default methods return always {\tt true};

    \item   {\tt cAlgo\_ParamVG} parametrize a weighted graph, this is done by  \emph{virtual} method
            {\tt tREAL8 WeightEdge()}, default methods return always {\tt 1.0} ;
            also {\tt cAlgo\_ParamVG} inherits from  {\tt cAlgo\_SubGr}.
\end{itemize}

For example, the method {\tt ShortestPath\_A2B}, from class {\tt cAlgoSP}, that compute the shortest path
between two vertices, takes as fourth parameter an object of type {\tt const cAlgo\_ParamVG \&}.


%   ------------------------------------------------------------------

\subsection{Connected component algorithms}

The class {\tt cAlgoCC} computes the connected component of a graph. It offers $3$ public services :

\begin{itemize}
    \item computation of the connected component of a single vertex, returns the component as vector of vertices;

    \item computation of the connected components of a set of vertice, returns the components as list of vector of vertices;

    \item computation of all the connected components of the graph, just a special case of previous one.
\end{itemize}

The $3$ methods take as last paramater an object of type {\tt const cAlgo\_SubGr \&}.


%   ------------------------------------------------------------------

\subsection{Mininum spaning tree, forest and shortest paths}

\subsubsection{class {\tt cAlgoSP}}

The class {\tt cAlgoSP} allows to compute the mininum spaning tree, forest and the shortest paths.
All three are done by the same class because algorithmically building a minimum spaning tree
is very close to building the tree of shortest path. 
The only difference is the agregation of cost; for curious readers, see the
method {\tt Internal\_MakeShortestPathGen} and observe there is only one extra 
line of code for the minimum spaning tree, this line is tagged by comment {\tt  @DIF:SP:MST} .


\subsubsection{Methods for shortest path}

There  exist  $2$ public methods for computing the shortest path :

\begin{itemize}
    \item {\tt ShortestPath\_A2B} the simpler and most common case,
          compute the shortest path between $2$ vertices,

    \item {\tt MakeShortestPathGen} compute the shortest path between $2$ sets of vertices.
\end{itemize}


\subsubsection{Methods for minimum spaning}

There exist $3$ methods for minimum spaning tree/forest :

\begin{itemize}
   \item {\tt MinimumSpanninTree} returns the minimum spaning tree containing a single vertex;

   \item {\tt MinimumSpanningForest} returns the minimum spaning forest of a set of vertices;

\end{itemize}


\subsubsection{Acces to results}

While doing the computation, the algorithm writes some fields inside the vertices.
This value written can be of interest :

\begin{itemize}
   \item  the method {\tt AlgoCost()}  allows to know the tree/path cost
   \item  the method {\tt BackTrackFathersPath}()  allows to extract the shortest path
          by back-tracking  the trees that were previously built.
\end{itemize}

Not very clear \dots and to complete by adding a way to memorize the vertices that have been reached.


%   ------------------------------------------------------------------
%   ------------------------------------------------------------------
%   ------------------------------------------------------------------

\section{The cycles-enumeration algorithm}

\label{Graph:AlgoEnumCycles}

%   ------------------------------------------------------------------

\subsection{Introduction}

The file {\tt MMVII\_Tpl\_GraphAlgo\_EnumCycles.h} contains class for computing all the
cycles of a graph $G$  of lenght inferior or equal to a given threshold $n$. 

Note that the problem is intrisically non polynomial because the size of the solution
is something like $\mathcal{O}(D^n)$ where $D$ is the average connectivity of $G$. So
obviously it can be called only with small values of $n$, whatever mean small (depending
of $D$, $card(G)$, user's patience \dots).  This said, the implementation we provide
is as efficient as it can be, which means that :


\begin{itemize}
   \item  has a cost no (much ?)  more than $Nb$ where $Nb$ is the size of the solution;
   \item  don't store explicitely all the computed cycles, which can be memory consuming, but 
          execute some user action when a cycle is found.
\end{itemize}

Also, there are is different possible definitions of what we call a cycle of size $n$, we can :

\begin{itemize}
   \item  accept or not that the cycle goes several times through the same vertex;
   \item  accept or not that the cycle goes several times through the same edge.
\end{itemize}

In our current implementation we accept the first but not the second.  Note that if required,
option  can be easily added to  extend to others possible definition.
The algorithm we use can be presented in two step : first, given a edge $E=\GrEdgAB$, compute all the
closed loop of size $\leq n$ which start from $v_2$ and ends to $v_1$.  Then define an algorithm
that compute all the cycle of graph.


%   ------------------------------------------------------------------

\subsection{Computing cycle going through a given edge}

\label{Graph:AlgoEC:OneEdge}

Basically, the algorithm is quite simple, it's just a recursive exlporation. Note
$Expl(P)$, the function that explorate all the path leading to $v_1$ ,  with the path $P =\{v_1,v_2,\dots,v\}$ ,
then the recursion is :

\begin{itemize}
    \item  start with $Expl(\{v_1,v_2\})$  ;
    \item  if $v_k=v_1$, we have cycle : do whatever must done with it (here call the user's callback method);
    \item  if $k\geq n$, return we are too far;
    \item  for all vertices $N^i(v_k)$ neighbours of $v_k$, call recursively  $Expl(\{v_1, \dots v_k,  N^i(v_k)\})$
\end{itemize}

This said, we must assure efficiently that we go only once throush each edge.  This is done by a marking
system \footnote{done with bit-flag allocation} such that :

\begin{itemize}
    \item we never explorate an edge marked;
    \item before exploring each edge of the neighbour, we mark it;
    \item after exploration we unmark it.
\end{itemize}

Another improvement is to avoid to go to far from $v_1$, at distance where there is no longer
hope to \emph{close the loop}. This is done this way, correctness of these "cuts" being
a consequence of \emph{triangular inequality} :

\begin{itemize}
   \item   note $d_G(A,B)$ the lenght of shortest path between $A$ and $B$ in $G$, it's a distance;

   \item  note $S(A,d)$ the \emph{sphere} defined by $ B \in S(A,d) \Leftrightarrow d_G(A,B) \leq d$;

   \item  before begining the recursion we compute $d_G(v_1,V)$ for all vertices in sphere $S_0 = S(v_1,\frac{n}{2})$ and mark them ,
           this can be done efficiently with shortest pathes methods described bellow;

   \item  in recursive exploration if $v_k \notin  S_0$ , dont explorate, we will never close the loop;

   \item  if $v_k \in  S_0$  but $k + d_G(v_1,v_k) > n$ , dont explorate, we will never close the loop.
\end{itemize}

%   ------------------------------------------------------------------

\subsection{Computing all the cycle}

\label{Graph:AlgoEC:AllC}

We could use naively the previous algorithm for each edge of the graph. But consider the cycle
$ABCD$, this naive implementation would extract $4$ times the cycle (begining by $AB$ then $BC$ \dots).
What we want, is to explorate once and only once this cycle. We start from the following basic remarks :

\begin{itemize}
   \item the cycle of lenght $n$ can be splited in two subsets : those that contain $AB$ and those
         that dont contains $AB$;

   \item the exploration of cycles containing $AB$ can be done using the algorithm defined in \ref{Graph:AlgoEC:OneEdge}

   \item once those computed, we can mark $AB$ as a \emph{forbiden} edge and those not containing $AB$ will be "naturally"
         explorated.
\end{itemize}

With these remarks,  the implementation for explorating all cycle is quite simple : for all edges $E$, explorated the cycle
going through $E$ taking into account previous marking, mark temporarilly $E$ as not usable.


%   ------------------------------------------------------------------

\subsection{Using the code}

The class implementing these algorithm are located in  {\tt MMVII\_Tpl\_GraphAlgo\_EnumCycles.h}. It mainly
defines two classes :

\begin{itemize}
    \item the class {\tt cActionOnCycle}, it's a pure virtual class, it's the base class for user's action
          to be executed each time a cycle is encountered;

    \item the class {\tt cAlgoEnumCycle}, that containt method to explorate all the cycles going through
          one edge, or all the cycles of the graph  (with length inferior to a given value);

     \item in the call back of an  {\tt cActionOnCycle}, the user receive the {\tt cAlgoEnumCycle} as parameter,
           which give access to the current path (and possibly later to others thing).
\end{itemize}

In the file {\tt BenchGraph.cpp} there is test on the correctness of the implementaion. It can be used
 as concrete  example of use of theses classes :

\begin{itemize}
   \item the class {\tt cCheckCycle}, that is a concrete derivate of  {\tt cActionOnCycle}, it action when called by
          {\tt OnCycle()}, is 
         \begin{itemize}
             \item test that the path is in fact a cycle
             \item accumulate information on all the cycles encountered : count them, accumulate a set of h-code on the cycles
         \end{itemize}

    \item the method  {\tt cBGG\_Graph::EnumCycle\_1Edge()} enumerate the cycle going trough an edge, and check that the number
          encountered is equal to some expected values;

    \item the method  {\tt cBGG\_Graph::Bench\_EnumCycle()} 
         \begin{itemize}
               \item   call previous one with some simple values (for which it was possible
                       to make the computation by hand);  
                \item also call the method {\tt ExplorateAllCycles()} and check that the cycles computed globally, 
                      using the marking system described in~\ref{Graph:AlgoEC:AllC} , are the same that those obtained
                       with individual computation.
         \end{itemize}

\end{itemize}


    % = = = = = = = = = = = = = = = = = = = = = = =

%\subsection{Heriting from  {\tt cMMVII\_Appli}}


%\subsubsection{Genarilities}


\chapter{Lidar-Image registration}
\label{Chap:LidImRed}


%-----------------------------------------------------------------------
%-----------------------------------------------------------------------
%-----------------------------------------------------------------------

\section{Introduction}

\subsection{Scope}

This chapter describe some draft idea for registration  between a Lidar
acquisition and image acquisition.  It describe not only the theory and equations,
but also some consideration about how concretely integrate it in \PPP.

The theoreticall hypothesis are :

\begin{itemize}
   \item  the lidar is the reference (it will not move) and that's the images localisation parameters
          we will  modify ; doing it in the reverse way would require
          a model of Lidar transformation, which is not obvious;

   \item  the images are already roughly registered, and what we want to do is a fine 
          registration (whatever it means, probably between $1$ and $10$ pixels, to see) ;
 
   \item  we work with vertical images (aerial or satellite),  the approach
          may be extended to terrestrial images, with few or no modification, but that's not our concern for now.
\end{itemize}

%-----------------------------------------------------------------------

\subsection{General framework}

The general idea of the proposed approach is :

\begin{itemize}
   \item  consider patches of lidar points (or individual lidar points in simplest variation) ;
   \item  consider for each patch a set of image where it is visible ;
   \item  consider the projection of the patch in each image;
   \item  define a \emph{radiometric} similarity  measurement between radiometric
          patches;
   \item  modify the images localisation parameters  for maximizing the global radiometric similarity between 
          projected patches.
\end{itemize}

%-----------------------------------------------------------------------

\subsection{Inclusion in MMVII bundle adjusment}

If there is sufficient relief, the maximization of similiarity of projected patches 
will be sufficient to get a unique solution. However it may be unstable on flat area (if we evaluate internal parameters),
or in area where there is few valuable patches (maybe in forest), and  by the way in our real
application we will have other measurement (GCP, GPS on centers, standard tie points \dots ) that we want to
integrate.

So the equation for similarity measurement will be considered as measures among others that will have
to be integrated in MMVII bundle adjustement. Part of this chapter will focus on this aspect,
mainly from practicle aspect, but also from theoreticall aspects (the classical question
of mixing equation of different nature).

%-----------------------------------------------------------------------

\subsection{Notation}
\label{LidImReg:Notation}

We have :

\begin{itemize}
    \item a central $3d$ lidar point $P$ ;
    \item possibly a set of $M$ neighouring $3d$ points  $p_i, i \in [1,M] $ of $P$;
    \item a set of $N$ images $I_k, k \in [1,N] $ such that $P$ and $p_i$ are visible in $I_k$;
    \item if $q$ is a $2d$ point, we note $I_k(q)$ the radiometry of $q$ in $I_k$,
          for now we consider only gray level image, the generalisation with multi-chanel
          image being obvious, and probably useless;
\end{itemize}

For each $I_k$ and each $3d$ point $P$ we note  $\pi_k(P)$ the projection of 
$P$ in $I_k$. $\pi_k$ depend from a certain number of parameters that will
be optimized in the bundle adjustment (rotation, pose, internal for perspective
camera; polynomial additionnal parameter for RPC).

Regarding the image, we will consider it as a function $\RR^2 \rightarrow  \RR$ :

\begin{equation}
    I_k : \RR^2 \rightarrow  \RR  ,  p    \rightarrow  I_k(p) 
\end{equation}

%-----------------------------------------------------------------------

\subsection{Image differentiation}

To use $I_k$ in iterative linearized least square minimizations,  we will need
to consider it as a differentiable function of $q$.  For this, for each point $q$,
we will assimilate locally $I_k$ to its Taylor expansion at point $q$. Let
note  $\vec G_k(q)$ the gradient of $I_k$ in $q$, we will write $I^q_k$ the
taylor expansion defined by :

\begin{equation}
   I^q_k(q') = I_k(q) + \vec G_k(q) \cdot  \overrightarrow{q q'} 
\end{equation}

Of course it's questionable, but we will do it \dots by the way, we think that the  high
redundance of measures will overcome this approximation.

For computing $I_k(q)$ and $\vec G_k(q)$, in a point of $\RR^2$,
we will select an interpolator and use the method {\tt GetValueAndGradInterpol}
defined in~\ref{Method:GetValueAndGradInterpol}.

In \PPP's jargon, for linearized system minimization, the values of $I_k$, $\vec G_k$   will be considered as \emph{observation} 
at the following steps :

\begin{itemize}
   \item  when adding an equation to the system with {\tt CalcAndAddObs}, simple case,
          or with {\tt AddEq2Subst}, case with Schurr's complement;
          
   \item  when generating the code for automatic differentiation in the method {\tt formula}
          of the class used for generating code (see~\ref{Class:Specif:Formulas}).
\end{itemize}

%-----------------------------------------------------------------------
%-----------------------------------------------------------------------
%-----------------------------------------------------------------------

\section{Simple formulations}

\label{LIR:SimplF}

%-----------------------------------------------------------------------

\subsection{Simplest case}

The first simplest solution is to consider the radiometric difference as
similarity measure.
Of course comparing directly the radiometry of images, may be too simplistic.
We know in image maching that it does not work with aerial image,  because natural
surface are not lambertian.   By the way , if the images
are radiometrically calibrated, which is more or less the case with professional
aerial images or satellite images, it may work statistically (i.e. say for $70\%$
of the points the radiometry are similar and for the other, the number of measure will
generate a non biased error that will average to $0$).


This modelisation suppose that there exist a radiometry 
common to all the images, we introduce this unknown radiometry as a 
temporary unknown $I^g$ .

Considering a point $P$, with notations of~\ref{LidImReg:Notation},
for each image $I_k$, we add  the equation :

\begin{equation}
    I_k(\pi_k(P)) = I^g \label{LIR:Eq1Point}
\end{equation}

Some detail about the implementation :

\begin{itemize}
    \item $I^g$ is an unknown, we must estimate its approximate value $I^g_0$,
          we can do it using the average of th $I_k(\pi_k(P))$ with
          current value of $\pi_k$;

    \item as there may exist million (billion ?) of equation, we dont
          want to have huge system, we will use Schurr complement to eliminate
          $I^g$ after processing point $P$;
\end{itemize}

Also it may be a problem to add directly equation~\ref{LIR:Eq1Point} in bundle
adjustment, because it will mix observation on radiometry with observation on
geometry.  We may prefer a version of~\ref{LIR:Eq1Point} with no dimension,
we can write :

\begin{equation}
    \frac{I_k(\pi_k(P))}{I^g} = 1
\end{equation}

Or if we want to avoid quotient in unknowns :

\begin{equation}
    \frac{I_k(\pi_k(P))}{I^g_0} = \frac{I_g}{I^g_0} 
\end{equation}

%-----------------------------------------------------------------------

\subsection{Image pre normalisation}

A possible way to  overcome the non lambertian aspect of images,
is to make a pre-processing of images, that make them localy invariant
to different transformation, for example :

\begin{itemize}
    \item to make it invariant to local scaling and translation, apply a Wallis filter : subsract the average and divide by the
          standard deviation (computed on given neighboorhood);

    \item   to make it simply invariant to local scaling ,  divided by local standard deviation, maybe more stable than
            Wallis.
\end{itemize}

Not sure if it will solve the problem, by the way it's easy to test.  
In a first approach these filtering, and many others, can be done with
the {\tt Nikrup}  command of {\tt mmv1}. And, if it turns to be the  "miracle solution",
a  filtering inside \PPP can be added later.  




%-----------------------------------------------------------------------

\subsection{Slightly more sophisticated similarity}

\label{LIR:CQ}

The next measure to test, is somewhat a variation on the well known
Census coefficient. My intuition is that in this context, it can be \emph{"the"} good compromize
simplicity/efficiency.

  %  -  -  -  -  -  -  -  -  -  -  -  -  -  -  -  -  -  -  -  -  -  -  -  -  -  -  -  -  -  -  -  -
\subsubsection{Quantitative Census}

Now we consider that we have no longer single pixels, but patches, and as
in Census, we use the central pixel as a normalisation factor. 
Noting $A = B$, between two boolean the function that values
$1$ if they have the same value en $0$ else, we define
Census, $C(V,W)$,  between two patches of radiometry $V_0, \dots$ and $W_0 \dots$ by :

\begin{equation}
      C(V,W)  = \sum_k (V_k > V_0) = (W_k>W_0)
\end{equation}


The drawback of census is  that because of boolean test $V_k > V_0$ , it'is
not continous and  (obvioulsy then) very sensitive to small variation and  difficutlt to derivate.
We make a first modification and define $C_{q0}$, a first quantitative version of census 
by :

\begin{equation}
      C_{q0} (V,W)  = \sum_k | \frac{V_k}{V_0} - \frac{W_k}{W_0} |
\end{equation}

Also the draw back of $C_{q0}$ is that  the ratio it is unbounded,
to overcome this problem we introduce the normalized ratio $\rho^n$ :

\begin{equation}
 \rho^n (x,y)  =
\left\{ \begin{array}{rcl}
\frac{x}{y} & \mbox{if}  & x<y\\
\\
2-\frac{y}{x} & \mbox{if}  & x \geq y
\end{array}\right.
\end{equation}

$\rho^n$ is a ratio function because $\rho^n(\lambda x, \lambda y) = \rho^n(x,y)$,
$\rho^n$ is bounded and  $C^1$ (but not $C^2$) . Many other choice would be possible, like \CPP's function {\tt atan2(x,y)} for
example who would be $C^{\infty}$; probably all the reasonable choice would be more or less equivalent, the advantage of $\rho^n$
is to be fast to compute.

We then define Census quantitative as :

\begin{equation}
      C_{q} (V,W)  = \sum_k | \rho^n(V_k,V_0) -  \rho^n(W_k,W_0)| \label{Eq:Census:Quant}
\end{equation}

  %  -  -  -  -  -  -  -  -  -  -  -  -  -  -  -  -  -  -  -  -  -  -  -  -  -  -  -  -  -  -  -  -
\subsubsection{Application to similarity measures}

The equation~\ref{Eq:Census:Quant} can be used directly for example in image matching.  
For the image-lidar registration we procede this way, using notation of~\ref{LidImReg:Notation} :


\begin{itemize}
    \item for $k\in [1,N] $ , the  $\pi_k(P) $ are the homologous central points;
    \item for $k\in [1,N], i \in [1,M]  $  ,the $\pi_k(p_i) $ are the homologous peripheral points;
\end{itemize}

If we had a perfect ratio conservation we should have :

\begin{equation}
     \forall i \in [1,M] :  \rho_i
                           =   \rho^n(I_1(\pi_k(P)), I_1(\pi_k(p_i)))
                           =  \rho^n(I_2(\pi_k(P)), I_2(\pi_k(p_i)))
                           \dots
                           =  \rho^n(I_N(\pi_k(P)), I_N(\pi_k(p_i)))
\end{equation}

Where $\rho_i$ is the theoretical common ratio between $P$ and $p_i$ 
projected radiometry in all images.
We proceed now as usual :


\begin{itemize}
    \item we introduce the $M$ temporary unknowns  $\rho_i  \in [1,M] $ , that represent
          the common ratio between radiometry of projections of $P$ and $p_i$ in all images;

    \item we estimate the initial value of $\rho_i$ by equation ~\ref{Eq:RIM:EstRho};

    \item we now add the $NM$ equation~\ref{Eq:RIM:EqRho} in the bundle adjsustment;

    \item and, as usual, we make a Schurr elimination of $M$ temporary unknowns.
\end{itemize}

\begin{equation}
     \rho_i = \frac{1}{N} \sum_k \rho^n(I_k(\pi_k(P)), I_k(\pi_k(p_i)))  \label{Eq:RIM:EstRho}
\end{equation}

\begin{equation}
      \forall i,k : \rho_i =  \rho^n(I_k(\pi_k(P)), I_k(\pi_k(p_i)))   \label{Eq:RIM:EqRho}
\end{equation}



%-----------------------------------------------------------------------

\section{Other stufs}

\subsection{Other similarity}
\label{LIR:OthSim}

To do later \dots , for example :

\begin{itemize}
    \item the "real" correlation coefficient, faisible with automatic differentiation,
          inconvenient we must generate a different formula for each number of points;

    \item adapt the structure of~\ref{LIR:CQ} to make efficient correlation :
          estimate a medium   patch of normalized radiometry in average and standard deviation,
          introduce  for each patch $2$ temporary unknowns, additive $A$ and multiplicative $B$,
          for each patch to be equal to the medium patch after transformation ;
          to think \dots

    \item and why not, test similarity coefficient from deep learning, as long as they can
          be differentiated, it can be injected in the proposed pipeline.
\end{itemize}

\subsection{Multi scale}
Maybe a multi-scale approach can be usefull when the initial localisation is not so good. A simple
way could be to just blurr the image, to be tested on simulated data ?

%-----------------------------------------------------------------------

\section{Correlation}

\subsection{Correlation formulation}


Let $X_i, i \in [1,N] $ be a series of values, noting :

\begin{equation}
     \mu(X) = \frac{\sum_{i=1}^N  X_i}{N}
\end{equation}

\begin{equation}
     \widetilde{X} = X - \mu(X)
\end{equation}

We have : 
\begin{equation}
     \mu(\widetilde{X}) = 0
\end{equation}

Noting :

\begin{equation}
     \sigma(X) =   \sqrt{\mu(\widetilde{X})^2)   }
\end{equation}

And  :
\begin{equation}
     \widehat {X} =   \frac{\widetilde{X}}{\sigma(X)}
\end{equation}

We have :

\begin{equation}
     \mu(\widehat {X}) =0    \; , \; \sigma(\widehat {X}) = 1  
    \label{CorrelSig1Avg0}
\end{equation}

and :

\begin{equation}
     X = \mu + \sigma \widehat {X}
    \label{CorrelXXHat}
\end{equation}



We remind that the normalized centered correlaction coefficient ($N_{CC}$)  between $2$ series $X_i$ and $Y_i$ is defined by :

\begin{equation}
    N_{CC}(X,Y) = \frac{\sum (X_i-\mu(X)) (Y_i-\mu(X)) }{\sigma(X)\sigma(Y)}   = \widehat {X} \cdot \widehat {Y}
\end{equation}

As $\widehat {X}$ and $\widehat {Y}$ are unitary vector ($\sigma = 1$), we have
$||\widehat {X} ,\widehat {Y}||^2  = 2 - 2 \widehat {X} \cdot \widehat {Y} $ and then :


\begin{equation}
    N_{CC}(X,Y) = 1-\frac{ D^2(\widehat {X} ,\widehat {Y}}{2}
\end{equation}

So maximizing $N_{CC}$ is equivalent to mininize $ D^2(\widehat {X} ,\widehat {Y}$. Consequently,
for using  $N_{CC}$, we will consider the problem of  solving the following equation of observation :

\begin{equation}
    \widehat {X} =\widehat {Y} \label{Eq:Cor:Pair:Norm}
\end{equation}

Consider now the case where we have $M$ series $X^j_i$ of $N$ values, $j \in [1,M], i \in [1,N]$
and that we want to modelize that they all are similar using correlation. We could use
equation $\ref{Eq:Cor:Pair:Norm}$ for all pair, but this would be un-elagant and un efficient.

Instead of that, we introduce the unknown serie $\chi_i$ which modelise the supposed common value
of all $\widehat X^j_i$. We the proceed this way :

\begin{itemize}
    \item as $\chi$ modelise the common value we impose series verifying equation~\ref{CorrelSig1Avg0}
          we impose $\mu(\chi)=0, \sigma(\chi)=1$;
    \item taking into account \ref{CorrelXXHat},  we introduce the unknowns $A^j,B^j$
          that link $\chi$ and $X^j$ with equation~\ref{Corel:AjBj} ;
\end{itemize}

\begin{equation}
	A^j X^j + B = \chi \label{Corel:AjBj}
\end{equation}

To summerize we try to solve the following system :

\begin{equation}
\left\{ \begin{array}{rc|l}
     \mu(\chi) = 0         \\ 
     \sigma(\chi) = 1         \\ 
      \forall i,j : \;  A^j X^j_i + B_j = \chi _i
\end{array}\right.
\label{XXXXX}
\end{equation}


In this system we have :

\begin{itemize}
	\item $2 M + N$ unknowns  $A^j,B^j,\chi _i$
   \item $2 + N*M$ equations;		 
\end{itemize}


For the initialisation, we can use for example equation  \ref{Correl:Chi0} for  $\chi$ .
For initialisation of $A^j, B^j$ we can use $\mu^j,\sigma^j$, or solve by least square the
equation ~\ref{Corel:AjBj}.

\begin{equation}
	\chi = \widehat {\frac{\sum \widehat {X^j} }{M}} \label{Correl:Chi0}
\end{equation}


%-----------------------------------------------------------------------

\section{Road map}

With MAC=Mohamed,

Some draft idea on the \emph{to do} list , and the  \emph{who do what} list :

\begin{itemize}
    \item  add a reader of lidar point, probably {\tt pdal}, in \PPP  :  {\bf MAC \& CM}

    \item  make a simulation of perfect data with blunder  (can blunder transformate its model in point cloud) :{\bf MAC \& JMM}

    \item make a firt implementation of~\ref{LIR:SimplF} to have a concrete example of 
         how to use  \PPP's library in this approach, test it with perfect data, check that  with
         a small perturbation of the orientation we are able to recover the ground truth :
         {\bf  MAC \& MPD} and {\bf  others} if interested to this new case of non linear optimisation library;

    \item test on real data equation ~\ref{LIR:SimplF}, then \ref{LIR:CQ} , then \ref{LIR:OthSim}, then \dots  : {\bf MAC} 
\end{itemize}

On real data an important issue will be  the selection of lidar point  : 

\begin{itemize}
     \item  obviously  avoid occluded point
     \item  probably   avoid vegetation, at least trees , maybe existing classif of lidar will be sufficient
     \item  probably there will far to many measures,  do we take all because it's not a time problem, 
            or do we make a random selection just to reduce time, or do we make a clever selection (textures \dots).
\end{itemize}






%%Different method 1 pixel, wallis filter   , cross pixel, census , correlation

%%Multi Scale
%%Pixel selection -> texture ? hidden ? relief ?
%  Road map ->  blender/JMM  , case 0 MPD/MALI ,  case 1, aerial : MALI 1pixel/walis , case 2 : 2 pixel

%\subsection{Generality}

  %  -  -  -  -  -  -  -  -  -  -  -  -  -  -  -  -  -  -  -  -  -  -  -  -  -  -  -  -  -  -  -  -
%\subsubsection{Circular target variant}


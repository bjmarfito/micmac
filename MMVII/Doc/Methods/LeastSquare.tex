\chapter{Leasy square and uncertainty estimation}
\label{Chap:LeastSquare}

%-----------------------------------------------------------------------
%-----------------------------------------------------------------------
%-----------------------------------------------------------------------

\section{Generality \& notations}

%-----------------------------------------------------------------------
\subsection{Notation}

We consider the system of $M$ equation with $N$ unkowns with $M\geq N$:

\begin{equation}
	\forall j : \sum_{i=1}^N l_{i,j} x_i = b_j
\end{equation}

Noting $L_j =  l_{i,j}$ and $X=x_i$ with $X$ and $L_j$ in $\RR^N$, we write :

\begin{equation}
	\forall j :  L_j . X = b_j \label{LstQ:EqLXb}
\end{equation}

%-----------------------------------------------------------------------
\subsection{Formalisation}

In general $M>N$ and there is no exact solution to equation\ref{LstQ:EqLXb},
so we replace the exact solving by the minimization of square residual $R(X)$ :


\begin{equation}
	R(X)  =  \sum_{j=1}^M (L_j \cdot X - b_j)^2
\end{equation}

Doing some algebric manipulation , using  $^t b= b$ for scalars and $X \cdot Y= ^t X Y = ^t Y X$ for vectors, we get :

\begin{equation}
\begin{array}{l}
	R(X)  \\
         =  \sum_j  (\trans X L_j -b_j)  ( \trans L_j  X -b_j) \\
	 = \trans X  (\sum_j L_j  \trans L_j) X - 2 (\sum_j L_j b_j) \cdot X + \sum_j b_j ^2
\end{array}
\end{equation}

Noting $A$ the \emph{normal} matrix and $B$ the ?? vector  with :

\begin{equation}
	A =   \sum_j L_j  \trans L_j  \; ; \;  B =\sum_j L_j b_j  \; ; \; C =  \sum_j b_j ^2
\end{equation}

We get :

\begin{equation}
	R(X) =  X \cdot A X - 2 B \cdot X +  C
\end{equation}


Noting $\overline X$ given by :

\begin{equation}
	\overline X = A^{-1} B \label{LstSq:Sol}
\end{equation}

And we can see that :

\begin{equation}
	R(X) =  (X-\overline X ) \cdot A  (X-\overline X )   +  C - B \cdot \overline X 
\end{equation}

The minimum of $R(X)$ is then reached for $X=\overline X$ and it is given by:

\begin{equation}
	R(\overline X) =    C - B \cdot \overline X \label{Lst:Res:Min}
\end{equation}


%-----------------------------------------------------------------------
%-----------------------------------------------------------------------
%-----------------------------------------------------------------------

\section{Values and uncertainty estimation}

%-----------------------------------------------------------------------
\subsection{Gauss-Markov theorem}
I we consider that each measure $L_j \cdot X - b_j$ is the realisation
of a random variable  such that :

\begin{itemize}
    \item each variable has an average of $0$ ;
    \item  the diffrente variable are independant \footnote{in fact un-correlated}
    \item  the different variable have  the same variance $\sigma ^2$.
\end{itemize}

Then \emph{Gauss-Markov} \footnote{see for example https://en.wikipedia.org/wiki/Gauss-Markov\_theorem}
theorem indicate that $\overline X$, as given in equation~\ref{LstSq:Sol}, is the 
best linear unbiased estimator of $X$ , and that $\overline X$ being considered as a random
variable, its variance-covariance $\Sigma$ can be computed by the un-biased estimator of
equation ~\ref{LstSq:EstVarCov} :

\begin{equation}
	\Sigma = A^{-1}  \frac{R(\overline X)}{N}  \frac{M}{M-N} = A^{-1} \sigma_0^2 \label{LstSq:EstVarCov}
\end{equation}

%-----------------------------------------------------------------------
\subsection{Efficiency consideration}

    %  -  -  -  -  -  -  -  -  -  -  -  -  -  -  -  -  -  -  -  -  -  -  -  -  -  -  -  -  -  -  -  -  -  -  -  -
\subsubsection{Case a single variable}

If we are interested in estimating the uncertainty of \emph{all} variables, we can
compute the matrix $A^{-1}$ and takes the diagonal element. By the way, with "big"
system this can be un-efficient in time and space; if are
interested in only one variable, say the $k^{th}$ :

\begin{itemize}
    \item  we only need to know the element $a^{-1}_{k,k}$
    \item  so noting $C^k$ the 
	   vector such that $C^k_i =\delta_k(i)$  (i.e the vector with all coordinates to $0$
           except  the $k^{th}$ to $1$);
    \item  it's sufficient to know $A^{-1} C^k$,
    \item  $A^{-1} C^k$ can computed by computing $C'^k$ such that $A C'^k = C^k$
    \item  depending on the structure of the sytem (sparsity \dots) and the used
	    algorithm (for example Cholesky) , solving \emph{only} equation $A C'^k = C^k$
		can be much more efficient than computing explicitely $A^{-1}$.
\end{itemize}

    %  -  -  -  -  -  -  -  -  -  -  -  -  -  -  -  -  -  -  -  -  -  -  -  -  -  -  -  -  -  -  -  -  -  -  -  -
\subsubsection{Case few variable}

Similarly, if we are interested in estimating the variance (and possibly covariance)  for
only a set $S$ of  few variable $S=\{k_1,\dots k_l\}$, noting the matrix $l \times N$
matrix $\mathcal{C}(S) = [C^{k_1} \dots C^{k_l}]$.

It is suffiient to compute the  matrix $ \mathcal{C}'(S)$ by
solving  the system $A \mathcal{C}'(S) = \mathcal{C}(S)$.

    %  -  -  -  -  -  -  -  -  -  -  -  -  -  -  -  -  -  -  -  -  -  -  -  -  -  -  -  -  -  -  -  -  -  -  -  -
\subsubsection{Case linear combination}

Some time, we are interested in the variance/vovariance  of  different linear combination.
Note $V_1, \dots ,V_l$ the $l$ vectors, such that  we want to estimate $V_1 \cdot X, \dots V_l \cdot X$.

For the value itself, due to linearity, the best estimator are simply $\overline{V_k} = V_k \cdot \overline X$.
For the variance/covariance, going back to equation~\ref{LstSq:EstVarCov}, we have :


\begin{equation}
	Cov(\overline{V_{k_1}},\overline{V_{k_2}})  =  \trans V_{k_1} A^{-1} V_{k_2}  \sigma_0^2 \label{LstSq:LinCombVarCov}
\end{equation}

And again \dots we dont need to compute $ A^{-1}$ but only to solve $A V'_{k_2}  = V_{k_2}$ and then use $V_{k_1} \cdot V'_{k_2}$.

    %  -  -  -  -  -  -  -  -  -  -  -  -  -  -  -  -  -  -  -  -  -  -  -  -  -  -  -  -  -  -  -  -  -  -  -  -
\subsubsection{Putting everything together}

\label{LstSq:Uncer:AllTog}

Note that to use formulas like~\ref{LstSq:LinCombVarCov}, we need to compute $\sigma_0^2$,  we need to compute $R(\overline(X))$,
and using equation~\ref{Lst:Res:Min}  we need to compute  $ \overline(X) =  A^{-1} B$.

So suppose that at a given step, we need to estimate the uncertainty of $l$ variables, $m$ linear combination , to minise
the computation cost, we will proceed this way :

\begin{itemize}
     \item  compute a $(1+l+m) \times N$ matrix $M$;
     \item  the first colum of $M$ will be $B$;
     \item  the $l$ next columns will be the $C_k$;
     \item  the $m$ next columns will be the $V_k$;
     \item  solve the  equation $A M' = M$;
     \item the first colum of $M'$ is $\overline X$, allow to compute $\sigma_0$
     \item next colums allow to compute the required variance/covariance for variables or their linear combination.
\end{itemize}


%-----------------------------------------------------------------------
%-----------------------------------------------------------------------
%-----------------------------------------------------------------------

\section{Uncertainty estimation in MMVII for the programmer}

%-----------------------------------------------------------------------
\subsection{When doing it ?} 

To do it  in a coherent way, the computation must be done when the matrixes are  complete, 
that is once we have added all the observations, all the constraints, eventually the
Levenberg-Markard "stabilisation".

But the moment where this is the case, is inside the method {\tt SolveUpdateReset}. If we do it before,
the system is not complete, if we do it after it will have been reseted and all the information will
be lost.  Of course these constraint are consequences of some implementation choices and could
have been avoided with some consequent re-organisation, which I did not want.

So to do it "at the good time", the way it is done is :

\begin{itemize}
   \item an optionnal structure must be passed to  {\tt SolveUpdateReset};
   \item this structure  will be filled inside the call to  {\tt SolveUpdateReset}.
\end{itemize}

%-----------------------------------------------------------------------
\subsection{The stucture {\tt cResult\_UC\_SUR}}

The structure {\tt cResult\_UC\_SUR} is the strcuture that will be passed,
its constructor contains :

\begin{itemize}
     \item a boolean indicating if we want to compute var/covar all the variable (just an easy way to fit $3^{rd}$ parameter);
     \item a boolean indicating if we want to compute normal matrix (generally useless)
     \item a list of indexes of variable for which we want to compute the var/covar
     \item a list of sparse vector for which we want to compute var/covar.
\end{itemize}

Inside the computation , the matrix $M$ descrined in ~\ref{LstSq:Uncer:AllTog} is computed
and the solution $M'$ of $AM' = M$ also.  After computation, what can be asked :


\begin{itemize}
      \item covariance between required variable
      \item covariance between required linear combination
      \item sigm0 (for test ?)
      \item access to $M'$ (just in case).
\end{itemize}


%-----------------------------------------------------------------------
\subsection{Correction test}

    %  -  -  -  -  -  -  -  -  -  -  -  -  -  -  -  -  -  -  -  -  -  -  -  -  -  -  -  -  -  -  -  -  -  -  -  -
\subsubsection{Motivation}
A relatively consequent test/bench of the implementation has been made. 
It can be found in {\tt Bench/BenchPropagUncert.cpp} . The motivation
of this bench is multiple :

\begin{itemize}
     \item check my understanding of the \emph{Gauss-Markov} theorem;
     \item check the correctness of the implementation
     \item check the  validity, or not, of \emph{Gauss-Markov} in some non
	     standard case (optimization with constraints, optimization with schur complement);
     \item potentially be used for didactic explanation of \emph{Gauss-Markov}
	     (but the code would require some simplification).
\end{itemize}

    %  -  -  -  -  -  -  -  -  -  -  -  -  -  -  -  -  -  -  -  -  -  -  -  -  -  -  -  -  -  -  -  -  -  -  -  -
\subsubsection{Formalization}

We have to generate data that are compliant with  gauss markov hypothesis. We do this way :

\begin{itemize}
     \item we generate $M$ random vector $V_1,V_2 \dots V_M$ of dimension $N$, that will correspond to 
           observations, assuring that they are not degenerate;

     \item we generate a random point $P$, the "solution" of the system

     \item for each observation  $V_k$, we will generate a discrete uniform law, centered on $V_k \cdot P$
	    on a small numer of values $n_k$ (typically between between $2$ and $5$) with the same variance $\sigma$
\end{itemize}

As the individual law of observation must be independant, all their combination are equi-probable. So
we generate all the $\mathcal{N} = n_1 * n_2 \dots * n_M$ possible combination of the indivual law.  For each
combination $c$  we compute :

\begin{itemize}
     \item compute the solution $S_c$  of the system;

     \item accumulate the $S_c$ in the computation of the \emph{empiricall} covariance matrix $\mathcal{C}$ of $S_c$;

     \item compute the estimator $\Sigma_c $ of the covariance for this combination, an accumulate it
	     to compute the average  $\overline \Sigma$;
\end{itemize}

More formally :

\begin{equation}
	\mathcal{C} =  \frac{\sum \trans S_c S_c }{\mathcal{N}} -   \frac{\trans \sum S_c}{\mathcal{N}}  \frac{\sum S_c}{\mathcal{N}}
\end{equation}

And  :

\begin{equation}
	\overline \Sigma =  \frac{\Sigma_c }{\mathcal{N}} 
\end{equation}


And if every thing works, then $ \Sigma$ is an unbiased estimator or $ \mathcal{C} $ and we must have :

\begin{equation}
	\overline \Sigma   = \mathcal{C}
\end{equation}

And it works, were can conclude that :

\begin{itemize}
     \item \emph{Gauss-Markov} is right, big news ;-) !
    \item even bigger news, the micmac implemetation is correct !
\end{itemize}

%\subsection{Uncertainty and constraint}
%\subsection{Uncertainty and Schurr complement}


